\documentclass[12pt,a4paper,english]{article}

% Packages
\usepackage[utf8]{inputenc}
\usepackage{amsmath}
\usepackage{amsfonts}
\usepackage{amssymb}
\usepackage{graphicx} % for figures
\usepackage{hyperref} % for hyperlinks
\usepackage{biblatex}
\addbibresource{references.bib}


\hypersetup{
	colorlinks=true, % Set to true to remove the border around links
	linkcolor=black,  % Color of internal links
	citecolor=blue,  % Color of citations
	filecolor=magenta, % Color of file links
	urlcolor=cyan,   % Color of URL links
	linktoc=all,      % 'all' will create links for both sections and page numbers in the TOC
}

\setlength{\parskip}{3mm plus 1mm minus 1mm}


% Document settings
\title{Toward a Generalized Framework for Stress in Materials: Moving Beyond the Cauchy Model}
\author{Foad S. Farimani}
\date{} % leave empty for the current date

\begin{document}

\maketitle

\begin{abstract}
The Cauchy stress tensor, a fundamental concept in classical continuum mechanics, has historically been assumed to be symmetric, often justified by applying the conservation of angular momentum to infinitesimal elements. While the validity of this symmetry assumption might appear settled, this paper explores a different perspective, demonstrating that the symmetry arises due to the assumed structure of the Cauchy stress field rather than being an inherent property of stress in materials. Specifically, we reveal that the irrotational nature of the Cauchy stress field inherently leads to the symmetry of the stress tensor, independent of the conservation of angular momentum.

Building upon this insight, the core of this paper delves into a critical analysis of the Cauchy stress tensor model. We argue that the linear dependence of stress on the surface is normal, as postulated in the Cauchy model, and may represent an oversimplification of the actual stress field in materials. This paper aims to establish a framework for moving beyond the Cauchy model by exploring the possibility of non-linear dependencies and their implications for constitutive behavior. While definitive proof of the limitations of the Cauchy model may necessitate further investigation, this work seeks to stimulate discussion and motivate the development of more generalized frameworks for describing stress in materials, ultimately contributing to a deeper understanding of material behavior within classical continuum mechanics.
\end{abstract}

% Keywords
\textbf{Keywords:} Cauchy stress tensor, continuum mechanics, material frame-indifference, objectivity, stress tensor symmetry, shear stress, non-linear stress, constitutive models, anisotropy

% Introduction
\section{Introduction}
% Background, motivation, and objectives of the research

The concept of stress within materials plays a central role in classical continuum mechanics.  It allows us to quantify the internal forces acting within a deformable body and forms the basis for understanding material behavior and predicting responses to external loads. The Cauchy stress tensor, introduced by Augustin-Louis Cauchy in the early 19th century, has become the cornerstone for representing stress in classical theory. However, despite its widespread use and success in numerous applications, certain assumptions and limitations of the Cauchy model warrant closer examination. 

This paper critically analyzes the Cauchy stress tensor and explores the potential for developing a more generalized framework for describing stress in materials. Our investigation stems from recognizing that the commonly accepted symmetry of the Cauchy stress tensor and its linear dependence on surface normals might not be inherent properties of stress but rather consequences of the specific structure and assumptions of the Cauchy model. 

% Subsections
\subsection{Stress in Classical Continuum Mechanics} % brief history and basic definitions 

Classical continuum mechanics provides a mathematical framework for analyzing the behavior of continuous materials, treating matter as a continuous distribution rather than considering the discrete nature of atoms and molecules. Within this framework, the concept of stress emerged as a means to quantify the internal forces acting within a deformable body \cite{truesdell2004non, marsden1994mathematical}. Unlike external forces that act on the body's surface, stress represents the internal distribution of forces acting across imaginary internal surfaces within the material.

Augustin-Louis Cauchy, a pioneer of continuum mechanics, formalized the concept of stress in the early 19th century by introducing the stress vector and the stress tensor \cite{truesdell2004non}. The stress vector, denoted as $\vec{\sigma}$, represents the force per unit area acting on a specific plane within the material at a given point. It depends on the location and the plane's orientation, defined by its normal vector $\hat{n}$. The stress tensor, a second-order tensor denoted as $\boldsymbol{\sigma}$, provides a complete description of the stress state at a point by encapsulating the stress vectors acting on all possible planes passing through that point.

The development of the Cauchy stress tensor and the associated theory laid the foundation for analyzing stress and deformation in solid and fluid materials. It allowed for the formulation of constitutive equations relating stress to strain and other kinematic variables, enabling the prediction of material responses under various loading conditions. However, as our understanding of material behavior has advanced, questions have arisen regarding the limitations of the Cauchy model and the need for more generalized frameworks.


\subsection{The Cauchy Stress Tensor and its Assumptions} % explain Cauchy's model and its underlying assumptions

The Cauchy stress tensor, denoted as $\boldsymbol{\sigma}$, is a cornerstone of classical continuum mechanics, serving to represent the state of stress at a point within a material. Its definition hinges on the concept of the stress vector, denoted as $\vec{\sigma}$, which quantifies the force per unit area acting on a plane with a specific unit normal vector $\hat{n}$ at a given point $\vec{r}$ \cite{truesdell2004non}. The stress vector is mathematically expressed as:

\begin{equation}\label{eq:stress_vector}
\vec{\sigma} = \boldsymbol{\sigma} \cdot \hat{n},
\end{equation}

\noindent
in the conventional form.

The Cauchy stress tensor is commonly concluded to be symmetric, implying that the shear stress components acting on orthogonal planes are equal, a concept often justified by applying the conservation of angular momentum to an infinitesimal cubic element of the material \cite{marsden1994mathematical}. However, as we will explore further in this paper, the symmetry of the Cauchy stress tensor can also be viewed as a consequence of the assumed structure of the stress field rather than an inherent property of stress itself. % not sure about this paragraph; we might be able to reach the conclusion without assuming a tensor form. #todo

In a 3D space, both the stress vector $\vec{\sigma}$ and the surface normal $\hat{n}$ can be represented as $3 \times 1$ matrices containing three rows and one column. To express equation \ref{eq:stress_vector} in matrix form, we can utilize the matrix representation of the stress tensor, often denoted as $\check{\sigma}$ or $\mathbf{\tau}$, leading to:

\begin{equation}\label{eq:matrix_form}
\mathbf{\sigma}_{3 \times 1} = \mathbf{\tau}_{3 \times 3} \, \mathbf{n}_{3 \times 1},
\end{equation}

The author's preference for avoiding the conventional tensor notation in this paper stems from concerns regarding the clarity and consistency of defining operations like the transpose for abstract tensor entities \cite{farimani2018mathstackexchange}. By using matrix and vector representations, we can maintain a more intuitive and unambiguous understanding of the mathematical operations involved.

Furthermore, from the perspective of differential geometry, the stress vector can be understood as a mapping from a five-dimensional space to a three-dimensional space. This five-dimensional space consists of the three coordinates defining the point $\vec{r}$ in the material and the two parameters needed to specify the orientation of the plane, represented by its unit normal vector $\hat{n}$. The stress vector then maps this point and orientation to the three components of the force per unit area acting on the plane. Conversely, the unit normal vector can be seen as a mapping from a two-dimensional space (parameter space for the plane's orientation) to the three-dimensional space of the vector components. This distinction highlights the stress field's complexity and the Cauchy model's potential limitations, which assume a simpler linear dependence on the surface normal, effectively reducing the problem's dimensionality.

\subsection{Motivation for a Generalized Framework} % discuss the limitations of the Cauchy model and the need for generalization

% I don't like the wording of this subsection. The "linear dependency" is the old way I was trying to explain the issue. I prefer to explain that oversimplifying the volume space with a 3D orthogonal differential element loses information.

The motivation for exploring a generalized framework for stress in materials within this research stems primarily from the limitations imposed by the dimensionality reduction inherent in the Cauchy model. By assuming a linear dependence of the stress vector on the surface normal, the Cauchy model effectively simplifies the complex nature of the stress field, potentially overlooking crucial information about the material's internal forces. As discussed earlier, the stress vector can be understood as a mapping from a five-dimensional space (point and plane orientation) to a three-dimensional space (force components). The linear dependence assumption of the Cauchy model reduces this five-dimensional space to a three-dimensional space represented by the stress tensor, possibly leading to an incomplete or inaccurate representation of the true stress state.

Investigating the possibility of non-linear dependencies between the stress vector and the surface normal could reveal a more comprehensive picture of the stress field within a material. This exploration aims to move beyond the limitations of the Cauchy model and develop a framework that accounts for the full dimensionality of the problem. By considering the potential for non-linear relationships, we may uncover a deeper understanding of the intricate interplay of internal forces within materials and pave the way for more accurate and sophisticated constitutive models.

While the issue of frame dependence and its connection to Material Frame-Indifference (MFI) has been a subject of debate in the continuum mechanics community, this research will not delve into the specifics of those discussions. Our focus remains on the potential limitations of the Cauchy model due to its dimensionality reduction and the exploration of a more generalized framework for representing stress. The observations of frame-dependent behavior in certain materials and flow regimes may or may not be relevant to the conclusions of this exploration, and further investigation is required to establish clear connections.

% Conservation Laws
\section{Conservation Laws in Vector Form}
% Derivation of conservation of mass and momentum for a control volume

The fundamental principles of conservation govern the behavior of physical systems, ensuring that certain quantities remain constant within a closed system or evolve predictably in the presence of external influences. In continuum mechanics, the conservation of mass and momentum are key principles guiding our understanding of material behavior. These laws dictate that the total mass and momentum within a control volume (a fixed region in space) must be conserved, accounting for the flux of these quantities across the control volume's boundaries and any external sources or forces acting on the material within.

This section delves into the derivation of the conservation laws for mass and momentum in their integral and differential forms using a control volume and the Eulerian description, which focuses on the properties and behavior of the material at fixed points in space. By expressing these laws in vector form and applying the divergence theorem, we establish the foundation for analyzing the local behavior of the continuum and set the stage for further exploration of the stress tensor and its properties.

% Subsections
\subsection{Conservation of Mass}

The principle of mass conservation states that the total mass within a closed system remains constant over time. In the context of continuum mechanics and fluid dynamics, this translates to the idea that the mass within a control volume can only change due to the net flow of mass across its boundaries. Let's consider a control volume $\forall$ with a closed surface boundary $\partial \forall$. The total mass $m$ contained within the control volume at any given time $t$ can be expressed as:

\begin{equation}
m = \int_{\forall} \rho(\vec{r}, t) d\forall,
\end{equation}

\noindent
where $\rho(\vec{r}, t)$ is the density of the material at point $\vec{r}$ and time $t$, and $d\forall$ represents a differential volume element.
The rate of change of mass within the control volume is given by the time derivative of the total mass:

\begin{equation}
\frac{dm}{dt} = \frac{d}{dt} \int_{\forall} \rho(\vec{r}, t) d\forall,
\end{equation}

The net mass flux across the control volume's boundary can be expressed as the surface integral of the mass flux density, which is the product of density and velocity $\vec{v}(\vec{r}, t)$:

\begin{equation}
\text{Mass flux} = -\oint_{\partial \forall} \rho(\vec{r}, t) \, \vec{v}(\vec{r}, t) \cdot d\vec{A},
\end{equation}

\noindent
where $d\vec{A}$ represents a differential area vector pointing outward from the control surface. The negative sign indicates that outward flow leads to a decrease in mass within the control volume.
Applying the principle of mass conservation, the rate of change of mass within the control volume must be equal to the net mass flux across its boundaries:

\begin{equation}\label{eq:mass_conservation_integral}
\frac{d}{dt} \int_{\forall} \rho(\vec{r}, t) d\forall = -\oint_{\partial \forall} \rho(\vec{r}, t) \, \vec{v}(\vec{r}, t) \cdot d\vec{A},
\end{equation}

This integral form of the mass conservation equation expresses the global balance of mass within the control volume. Using the Gauss-Ostrogradsky divergence theorem, which relates the flux of a vector field through a closed surface to the divergence of the field within the enclosed volume, we can convert the integral form of the mass conservation equation (equation \ref{eq:mass_conservation_integral}) into a partial differential equation (PDE) representing the local mass balance at each point within the continuum. The divergence theorem states:

\begin{equation}
\oint_{\partial \forall} \vec{F} \cdot d\vec{A} = \int_{\forall} (\nabla \cdot \vec{F}) d\forall,
\end{equation}

\noindent
where $\vec{F}$ is a vector field and $\nabla \cdot \vec{F}$ is its divergence. Applying the divergence theorem to the mass flux term in equation \ref{eq:mass_conservation_integral}:

\begin{equation}
-\oint_{\partial \forall} \rho(\vec{r}, t) \vec{v}(\vec{r}, t) \cdot d\vec{A} = -\int_{\forall} \nabla \cdot (\rho(\vec{r}, t) \, \vec{v}(\vec{r}, t)) d\forall,
\end{equation}

\noindent
Substituting this into equation \ref{eq:mass_conservation_integral} and bringing the time derivative inside the integral (assuming a fixed control volume):

\begin{equation}
\int_{\forall} \left( \frac{\partial \rho(\vec{r}, t)}{\partial t} + \nabla \cdot (\rho(\vec{r}, t) \, \vec{v}(\vec{r}, t)) \right) d\forall = 0,
\end{equation}

\noindent
Since this equation must hold for any arbitrary control volume $\forall$, the integrand itself must be zero at every point within the continuum:

\begin{equation}\label{eq:mass_conservation_pde}
\frac{\partial \rho(\vec{r}, t)}{\partial t} + \nabla \cdot (\rho(\vec{r}, t) \, \vec{v}(\vec{r}, t)) = 0,
\end{equation}

\noindent
This is the continuity equation, the PDE form of the mass conservation law, which expresses the local balance of mass at each point in the continuum.

\subsection{Conservation of Linear Momentum}

The principle of linear momentum conservation states that the rate of change of momentum of a system is equal to the net force acting on the system. In continuum mechanics, this principle, when applied to a control volume, relates the rate of change of momentum within the volume to the surface forces acting on its boundary, the body forces acting on the material within the volume, and the flux of momentum across the control surface due to the movement of material. Let's consider a control volume $\forall$ with a closed surface boundary $\partial \forall$. The total linear momentum $\vec{P}$ of the material within the control volume at any time $t$ can be expressed as:

\begin{equation}
\vec{P} = \int_{\forall} \rho(\vec{r}, t) \vec{v}(\vec{r}, t) d\forall,
\end{equation}

\noindent
The rate of change of linear momentum is given by the time derivative of the total momentum:

\begin{equation}
\frac{d\vec{P}}{dt} = \frac{d}{dt} \int_{\forall} \rho(\vec{r}, t) \vec{v}(\vec{r}, t) d\forall,
\end{equation}

\noindent
The surface forces acting on the boundary of the control volume can be represented by the stress vector $\vec{\sigma}(\vec{r}, t, \hat{n})$ integrated over the surface:

\begin{equation}
\text{Surface forces} = \oint_{\partial \forall} \vec{\sigma}(\vec{r}, t, \hat{n}) dA,
\end{equation}

\noindent
where $\hat{n}$ is the outward unit normal vector at each point on the surface. The body forces acting on the material within the control volume can be represented by the force per unit volume $\vec{b}(\vec{r}, t)$ integrated over the volume:

\begin{equation}
\text{Body forces} = \int_{\forall} \vec{b}(\vec{r}, t) d\forall,
\end{equation}

\noindent
% Applying the principle of linear momentum conservation:

% \begin{equation}\label{eq:momentum_conservation_integral}
% \frac{d}{dt} \int_{\forall} \rho(\vec{r}, t) \vec{v}(\vec{r}, t) d\forall = \oint_{\partial \forall} \vec{\sigma}(\vec{r}, t, \hat{n}) dA + \int_{\forall} \rho(\vec{r}, t) \vec{b}(\vec{r}, t) d\forall,
% \end{equation}

\noindent
The flux of momentum across the control surface arises from the movement of material carrying momentum with it. This can be expressed as:

\begin{equation}
\text{Momentum flux} = - \oint_{\partial \forall} \rho(\vec{r}, t) \vec{v}(\vec{r}, t) (\vec{v}(\vec{r}, t) \cdot d\vec{A}),
\end{equation}

The negative sign indicates that the outward flow of momentum leads to decreased momentum within the control volume.

Applying the principle of linear momentum conservation, the rate of change of momentum within the control volume must be equal to the net effect of surface forces, body forces, and momentum flux:

\begin{equation}\label{eq:momentum_conservation_integral}
\begin{split}
\frac{d}{dt} \int_{\forall} \rho(\vec{r}, t) \vec{v}(\vec{r}, t) d\forall = \oint_{\partial \forall} \vec{\sigma}(\vec{r}, t, \hat{n}) dA \\
+ \int_{\forall} \vec{b}(\vec{r}, t) d\forall - \oint_{\partial \forall} \rho(\vec{r}, t) \vec{v}(\vec{r}, t) (\vec{v}(\vec{r}, t) \cdot d\vec{A})
\end{split}
\end{equation}

\noindent
This integral form of the momentum conservation equation represents the global balance of momentum within the control volume.

\subsubsection{Challenges in Conversion to PDE Form}

Converting the stress vector's surface integral to a volume integral using the divergence theorem or other relevant theorems is not straightforward due to the stress vector's dependence on both position and surface orientation. Existing theorems like the divergence theorem or Stokes' theorem typically deal with vector fields that depend solely on position. Finding a suitable generalization or alternative approach for converting the surface integral is a key challenge that requires further investigation.

As highlighted in the previous paragraph, directly converting the surface integral of the stress vector in the momentum conservation equation (equation \ref{eq:momentum_conservation_integral}) to a volume integral using standard theorems like the divergence theorem or Stokes' theorem is not feasible. The primary obstacle lies in the dependence of the stress vector $\vec{\sigma}(\vec{r}, t, \hat{n})$ on both the position vector $\vec{r}$ and the surface normal vector $\hat{n}$. Existing integral transformation theorems typically deal with vector fields that depend solely on position, making them inadequate for handling the additional dependence on surface orientation present in the stress vector.

This challenge suggests that the crux of the Cauchy model and its potential limitations may lie in simplifying the control volume to an infinitesimal orthogonal element. By assuming an orthogonal shape, the dependence of the stress vector on the surface orientation is effectively ignored, allowing for the application of the divergence theorem and leading to the conclusion of a symmetric stress tensor. However, this simplification overlooks the complex five-dimensional nature of the stress field and may not accurately represent the true behavior of stress in materials. This observation motivates a deeper exploration of the Cauchy stress tensor and its assumptions, paving the way for developing a more generalized framework to address these limitations.

% \subsection{Conservation of Angular Momentum} % (Optional)

% Critical Analysis of Cauchy Stress Tensor
\section{Limitations of the Cauchy Stress Tensor}
% Discussion of the issues with Cauchy's assumptions and model

As discussed in the previous sections, the Cauchy stress tensor and its associated theory have served as a cornerstone for stress analysis in classical continuum mechanics. However, the assumptions underlying the Cauchy model, particularly those related to the control volume and the linear dependence of stress on surface normals, raise concerns about the model's ability to capture the complexities of stress in materials fully.

This section delves deeper into the limitations of the Cauchy stress tensor by examining the mathematical disconnect between Cauchy's method of using an infinitesimal orthogonal differential element and the general form of the control volume assumed in the conservation laws. We will highlight how this simplification overlooks the multi-dimensional nature of the stress field and potentially leads to an incomplete or inaccurate representation of the stress state within a material.

% Subsections
\subsection{Dimensionality Reduction and the Control Volume}

While the Cauchy stress tensor provides a valuable tool for stress analysis, it relies on a critical assumption that may lead to an oversimplification of the stress field within a material. This assumption involves simplifying the control volume to an infinitesimal element with an orthogonal shape, often considered a cuboid.

As discussed previously, the stress vector $\vec{\sigma}(\vec{r}, t, \hat{n})$ depends not only on the position vector $\vec{r}$ but also on the orientation of the plane, defined by its unit normal vector $\hat{n}$. This implies that the stress vector is inherently a function of five independent variables: three coordinates for the position and two parameters to specify the plane's orientation. However, by assuming an orthogonal control volume, Cauchy's approach effectively reduces the problem's dimensionality. The orthogonal shape implies that the surface normals of the control volume's faces are fixed and known, eliminating the dependence of the stress vector on the plane orientation. This simplification allows for applying the divergence theorem and leads to deriving the local form of the momentum balance equation with a symmetric stress tensor.

However, this dimensionality reduction potentially overlooks crucial information about the stress field's true nature. The five-dimensional essence of the stress vector is lost in the process, and the resulting Cauchy stress tensor may only provide a limited representation of the stress state within the material. This observation motivates a deeper investigation into the consequences of the linear stress assumption and the exploration of more generalized frameworks that can account for the full dimensionality of the problem.

\subsection{Consequences of Linear Stress Assumption}

A key assumption within the Cauchy model is the linear dependence of the stress vector on the surface normal. This means that the stress vector acting on any plane passing through a point can be obtained by taking the dot product of the Cauchy stress tensor $\boldsymbol{\sigma}$ with the unit normal vector $\hat{n}$ of the plane, as expressed in equation \ref{eq:stress_vector}. While this assumption simplifies the mathematical formulation and enables the application of the divergence theorem, it also leads to a significant consequence: the inherent symmetry of the Cauchy stress tensor.

As discussed in the previous section, simplifying the control volume to an infinitesimal orthogonal element allows for the conversion of the surface integral of the stress vector into a volume integral using the divergence theorem. This leads to deriving the local form of the momentum balance equation with a symmetric stress tensor. However, the symmetry arises not from the conservation of angular momentum itself but rather as a direct consequence of the assumed linear dependence of stress on the surface normal. This linearity implies that the stress field is irrotational, meaning its curl is zero. The irrotational nature of the stress field inherently results in the symmetry of the stress tensor, independent of the conservation of angular momentum.

This observation suggests that the symmetry of the Cauchy stress tensor might not be a fundamental property of stress in materials but rather a consequence of the specific assumptions made in the Cauchy model. This raises questions about the universality of the Cauchy model and motivates the exploration of more generalized frameworks that can account for the possibility of non-linear stress dependencies and potentially non-symmetric stress tensors.

\subsection{Analysis of Cauchy's Calculations}

To illustrate how the stress tensor's symmetry emerges from the Cauchy model's assumptions, let's examine Cauchy's classical calculations using a simplified control volume. Consider an infinitesimal cubic element with sides of length $dx$, $dy$, and $dz$ aligned with the Cartesian coordinate axes. The stress vectors acting on the faces of this element can be expressed using equation \ref{eq:stress_vector} and the known unit normal vectors of the cube faces (e.g., $\hat{n} = \hat{x}$ for the face perpendicular to the x-axis).

By applying the conservation of angular momentum to this cubic element and taking the limit as the side lengths approach zero, one obtains the following relationship between the shear stress components:

\begin{equation}
\sigma_{xy} = \sigma_{yx}, \quad \sigma_{xz} = \sigma_{zx}, \quad \sigma_{yz} = \sigma_{zy},
\end{equation}


This demonstrates the symmetry of the stress tensor. However, it is crucial to recognize that this result relies on two key assumptions: (1) the linear dependence of the stress vector on the surface normal and (2) the use of an infinitesimal cubic element as the control volume. As we have discussed, the first assumption leads to an irrotational stress field, which inherently results in the symmetry of the stress tensor. The second assumption, by fixing the surface normals, eliminates the dependence of the stress vector on the plane's orientation and allows for the application of the divergence theorem.

Therefore, the symmetry of the Cauchy stress tensor emerges not as a fundamental law but as a consequence of the specific model and its simplifying assumptions. This motivates the exploration of more generalized frameworks that can account for the full complexity of the stress field and the possibility of non-symmetric stress tensors in materials.


\section{Challenging the Cauchy Paradigm: Exploring Alternative Formulations}

The previous sections highlighted the potential limitations of the Cauchy stress tensor model, particularly concerning its dimensionality reduction and the assumptions associated with the control volume and linear stress dependence. This section aims to investigate these limitations further by exploring alternative formulations and approaches within the realm of classical continuum mechanics. Our objective is to either provide concrete counter-examples that demonstrate the inadequacy of the Cauchy model or identify pathways to circumvent the challenges associated with converting the integral form of momentum conservation into its partial differential equation (PDE) form without resorting to oversimplifying assumptions.

To achieve this, we will investigate various strategies, including analyzing stress in different coordinate systems, employing alternative mathematical representations, and exploring different formulations of classical mechanics. By examining these diverse approaches, we seek to gain a deeper understanding of the complexities of stress in materials and pave the way for developing more general and accurate models.

\subsection{2D Cartesian Coordinate Systems with Different Orientations}

To investigate the potential dependence of the stress tensor on the choice of the coordinate system, we will analyze the Cauchy stress model in the context of two-dimensional Cartesian coordinate systems with different orientations. By comparing the stress tensors obtained from these rotated systems, we can assess whether they are trivially equal, as implied by the Cauchy model, or if discrepancies arise that challenge the generality of the model.

This analysis will focus on a simplified 2D scenario to enhance clarity and facilitate visualization of the concepts involved. We will begin by establishing the transformation rules for the stress vector and the surface normal under coordinate system rotations. Subsequently, we will examine the resulting transformation of the Cauchy stress tensor and explore its implications for the model's validity and limitations.

\subsubsection{Coordinate System Rotation and Transformation Rules}



\subsubsection{Transformation of the Cauchy Stress Tensor}
\subsubsection{Analysis and Implications}

\subsection{2D Polar Coordinates and Complex Representation}

\subsection{3D Rotated Cartesian Coordinate Systems}

\subsection{3D Cylindrical and Spherical Coordinates with Quaternion Representation (Optional)}

\subsection{Alternative Classical Mechanics Formulations: Lagrangian and Hamiltonian Approaches (Optional)}

% Conclusions and Future Work
\section{Conclusions and Future Directions}
% Summary of findings and suggestions for further research

% References
\printbibliography

\end{document}